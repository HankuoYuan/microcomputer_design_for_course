\section{硬件设计}
HL-1 单片机实验板原理图如\autoref{HL-1开发板原理图}所示。
\begin{figure}[!htbp]
    \centering
    \includegraphics[width =0.9\textwidth]{figures/开发板原理图.png}
    \caption{HL-1单片机实验板原理图}
    \label{HL-1开发板原理图}
\end{figure}
\newline

HL-1智能小车原理图如\autoref{HL-1智能小车原理图}所示。
\begin{figure}[!htbp]
    \centering
    \includegraphics[width =0.9\textwidth]{figures/智能小车原理图.png}
    \caption{HL-1智能小车原理图}
    \label{HL-1智能小车原理图}
\end{figure}
\newline

HL-1智能小车和51开发板接线如下:

\textbf{J3}
\begin{itemize}
    \item IN1--接到--实验板上的P1.2
    \item IN2--接到--实验板上的P1.3
    \item EN1--接到--实验板上的P1.4
    \item EN2--接到--实验板上的P1.5
    \item IN3--接到--实验板上的P1.6
    \item IN4--接到--实验板上的P1.7
\end{itemize}

\textbf{J4}
\begin{itemize}
    \item IN5--接到--实验板上的P2.1
    \item IN6--接到--实验板上的P2.0
\end{itemize}

\textbf{J5}
\begin{itemize}
    \item OUT1--接到--实验板上的P3.3
    \item OUT2--接到--实验板上的P3.4
    \item OUT3--接到--实验板上的P3.5
    \item OUT4--接到--实验板上的P3.6
\end{itemize}

\textbf{实验板供电接线}

\textbf{J5}
\begin{itemize}
    \item VCC--接到--实验板上的VCC引脚
    \item GND--接到--实验板上的GND引脚
\end{itemize}
\begin{enumerate}
    \item J6是超声波模块插座
    \item J7是蓝牙模块插座
    \item J8是蓝牙信号接到开发板的P3.0和P3.1
    \item J17是车底盘四路5V对外供电接口
\end{enumerate}

\textbf{风扇模块}
\begin{itemize}
    \item GND接到车底盘J17接口的GND引脚
    \item VCC接到车底盘J17接口的VCC引脚
    \item IN引脚接到火焰传感器的S引脚
\end{itemize}

\textbf{火焰传感器}
\begin{itemize}
    \item GND接到车底盘J17接口的GND引脚
    \item VCC接到车底盘J17接口的VCC引脚
    \item  S 引脚接到风扇模块的IN引脚
\end{itemize}

其中火焰传感器在没有检测到火焰时S引脚输出高电平,检测到火焰则输出低电平;而风扇模块的IN引脚在收到低电平时启动风扇,高电平则关闭风扇。这两个模块结合在一起刚好可以自动检测火焰并吹灭,不需要单片机加以任何控制。









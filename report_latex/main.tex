%!TeX program = xelatex
\documentclass[12pt,a4paper,UTF8]{ctexart}
\usepackage{swjtuReport}
% 将一级标题改成“第1章”,若\arabic改成\chinese即为“第一章”
% \usepackage{titletoc}
% \ctexset{section={name={第,章},number={\arabic{section}}}}   

\begin{document}
%%------------------------------封面------------------------------%%
\cover
\thispagestyle{empty} % 首页不显示页码
%%------------------------------摘要------------------------------%%
\newpage
\pagenumbering{Roman} % 重新计算页码,并将摘要、目录的页码设置成罗马数字页码

\begin{abstract}\normalsize   % 更改摘要的内容的字体大小

本课程设计报告详细介绍了“小车远程控制系统设计与实现”项目,该项目旨在通过蓝牙模块或红外遥控器实现智能小车的远程控制,涵盖行走、倒退、转弯等基本功能,同时利用LED灯或显示器件显示小车状态。报告首先概述了智能小车远程控制系统的当前技术状况,包括蓝牙、红外、Wi-Fi以及4G/5G等控制技术的应用,并探讨了LED显示与状态反馈、个性化功能拓展如自动避障和AI集成的发展趋势。硬件设计部分展示了HL-1单片机实验板与智能小车的原理图及接线方案,软件设计部分涉及代码调试与仿真。调试结果显示,小车能成功响应远程指令并执行相应动作。课设问题分析部分指出了小车硬件的局限性,提出了改进意见和未来研究方向,包括集成AI与机器学习、AR/VR技术、无线充电、物联网与大数据分析,以及强化安全性和隐私保护措施。最后,报告强调了智能小车对教育、健康、法律和文化的广泛影响,预示了其在社会生活和工业生产中的应用前景。

\noindent \textbf{关键词:}
微机原理;51单片机;智能小车;蓝牙模块;LCD屏显示

\end{abstract}
%%------------------------------目录页----------------------------%%
\newpage
% \pagenumbering{Roman} % 重新计算页码,并将摘要、目录的页码设置成罗马数字页码
\begin{center}
    \tableofcontents
\end{center}
%%------------------------------正文------------------------------%%
\newpage
\pagenumbering{arabic} % 重新计算页码,并将正文的页码设置成阿拉伯数字页码

\section{考核任务书}
\subsection{课题要求}
\textbf{小车远程控制系统设计与实现}
\begin{itemize}
    \item 利用蓝牙模块或红外遥控器实现智能小车基本的行走倒退转弯功能。
    \item 利用显示器件或LED灯实时显示小车状态。
    \item 在完成基本操作的基础上,鼓励扩展个性化功能。
\end{itemize}
\subsection{个人拓展部分}
我加了一个灭火功能,该功能硬件由风扇模块和火焰传感器组成,不需要单片机控制,可以实现自动检测火焰并吹灭,具体接线可以看\autoref{硬件设计}关于风扇模块和火焰传感器的部分。
\subsection{课程设计要求}
综合运用Proteus/Keil软件构建单片机外部扩展系统,能够完成从原理图设计、代码调试到单片机与外围电路协同仿真、调试,并在仿真调试的基础上,实现单片机智能小车以及其他功能模块的控制系统开发,熟练掌握单片机系统设计以及项目开发调试的主要流程与方法。




\section{背景调研}
\subsection{智能小车远程控制系统的现状与发展趋势}
\subsubsection{国内外发展现状}

\textbf{控制技术}
\begin{itemize}
    \item 蓝牙与红外控制:蓝牙和红外遥控技术在智能小车领域的应用已相当成熟。蓝牙提供了相对稳定的无线连接,支持长距离(理论上可达100米)和高数据传输速率,适合于需要实时反馈和控制的场景。红外遥控则因其成本低、功耗小的特点,在短距离控制中占有优势。
    \item Wi-Fi与4G/5G控制:随着物联网技术的发展,Wi-Fi和4G/5G网络也被广泛应用于智能小车的远程控制。Wi-Fi提供更远的通信距离和更高的数据传输速度,而5G网络更是以其超低延迟和高速度特性,为智能小车的远程实时控制开辟了新的可能。
\end{itemize}
\textbf{显示与状态反馈}
\begin{itemize}
    \item LED显示:LED灯因其能耗低、响应快、寿命长等优点,常被用作智能小车的状态指示。通过编程控制,LED灯可以显示小车的工作状态、电量信息等,增强人机交互体验。
    \item 显示屏集成:更高阶的系统可能会集成LCD或OLED显示屏,不仅能显示实时状态,还能呈现更复杂的信息,如地图导航、传感器数据等。
\end{itemize}
\textbf{个性化功能拓展}
\begin{itemize}
    \item 自动避障与路径规划:通过集成超声波传感器、红外传感器或摄像头,智能小车能够感知周围环境,实现自动避障和路径规划,提高自主性。
    \item 人工智能集成:深度学习、机器视觉等AI技术的应用,使智能小车具备识别目标、理解环境的能力,可用于搜救、监控等多种场景。
    \item 多车协同作业:通过网络连接,多辆智能小车可以协同工作,执行复杂的任务,如物流配送、农业耕作等。
\end{itemize}
\subsubsection{发展趋势}
\begin{itemize}
    \item 技术融合:未来智能小车将更加依赖于多技术融合,包括但不限于物联网、大数据、云计算和人工智能,以提升其智能水平和适应性。
    \item 标准化与模块化:随着技术成熟,智能小车的设计趋向于标准化和模块化,便于快速定制和升级,满足不同应用场景的需求。
    \item 安全性与隐私保护:随着智能小车在个人和商业领域的广泛应用,安全性和隐私保护将成为关键技术挑战之一,需要开发更高级别的加密技术和安全协议。
    \item 能源效率与可持续性:能源效率和环保意识的提升将推动智能小车采用更高效的动力系统和可再生能源,如太阳能充电。
\end{itemize}

综上所述,智能小车远程控制系统的研发正朝着智能化、高效化、安全化的方向发展,未来将更紧密地融入人们的日常生活和工业生产中。






\section{硬件设计}
HL-1 单片机实验板原理图如\autoref{HL-1开发板原理图}所示。
\begin{figure}[!htbp]
    \centering
    \includegraphics[width =0.9\textwidth]{figures/开发板原理图.png}
    \caption{HL-1单片机实验板原理图}
    \label{HL-1开发板原理图}
\end{figure}
\newline

HL-1智能小车原理图如\autoref{HL-1智能小车原理图}所示。
\begin{figure}[!htbp]
    \centering
    \includegraphics[width =0.9\textwidth]{figures/智能小车原理图.png}
    \caption{HL-1智能小车原理图}
    \label{HL-1智能小车原理图}
\end{figure}
\newline

HL-1智能小车和51开发板接线如下:

\textbf{J3}
\begin{itemize}
    \item IN1--接到--实验板上的P1.2
    \item IN2--接到--实验板上的P1.3
    \item EN1--接到--实验板上的P1.4
    \item EN2--接到--实验板上的P1.5
    \item IN3--接到--实验板上的P1.6
    \item IN4--接到--实验板上的P1.7
\end{itemize}

\textbf{J4}
\begin{itemize}
    \item IN5--接到--实验板上的P2.1
    \item IN6--接到--实验板上的P2.0
\end{itemize}

\textbf{J5}
\begin{itemize}
    \item OUT1--接到--实验板上的P3.3
    \item OUT2--接到--实验板上的P3.4
    \item OUT3--接到--实验板上的P3.5
    \item OUT4--接到--实验板上的P3.6
\end{itemize}

\textbf{实验板供电接线}

\textbf{J5}
\begin{itemize}
    \item VCC--接到--实验板上的VCC引脚
    \item GND--接到--实验板上的GND引脚
\end{itemize}
\begin{enumerate}
    \item J6是超声波模块插座
    \item J7是蓝牙模块插座
    \item J8是蓝牙信号接到开发板的P3.0和P3.1
    \item J17是车底盘四路5V对外供电接口
\end{enumerate}

\textbf{风扇模块}
\begin{itemize}
    \item GND接到车底盘J17接口的GND引脚
    \item VCC接到车底盘J17接口的VCC引脚
    \item IN引脚接到火焰传感器的S引脚
\end{itemize}

\textbf{火焰传感器}
\begin{itemize}
    \item GND接到车底盘J17接口的GND引脚
    \item VCC接到车底盘J17接口的VCC引脚
    \item  S 引脚接到风扇模块的IN引脚
\end{itemize}

其中火焰传感器在没有检测到火焰时S引脚输出高电平,检测到火焰则输出低电平;而风扇模块的IN引脚在收到低电平时启动风扇,高电平则关闭风扇。这两个模块结合在一起刚好可以自动检测火焰并吹灭,不需要单片机加以任何控制。









\section{软件设计}
\textbf{Keil5的main.c文件源码,相关注释已在代码中:}
\lstinputlisting[language=C]{docs/main.c}
\section{调试结果}
\subsection{仿真调试}
IO口仿真调试图如\autoref{IO口仿真调试图}所示:
\begin{figure}[!htbp]
    \centering
    \includegraphics[width =0.9\textwidth]{figures/io口仿真.png}
    \caption{IO口仿真调试图}
    \label{IO口仿真调试图}
\end{figure}
\newline
串口仿真调试图如\autoref{串口仿真调试图}所示:
\begin{figure}[!htbp]
    \centering
    \includegraphics[width =0.9\textwidth]{figures/串口仿真.png}
    \caption{串口仿真调试图}
    \label{串口仿真调试图}
\end{figure}
\newline
LCD仿真调试以小车实物为调试平台,调试结果如\autoref{forward}-\autoref{stop}所示:
\begin{figure}[htbp]
	\centering
	\begin{minipage}{0.49\linewidth}
		\centering
		\includegraphics[width=0.9\linewidth]{figures/forward.jpg}
		\caption{前进LCD状态显示图}
		\label{forward}%文中引用该图片代号
	\end{minipage}
	\begin{minipage}{0.49\linewidth}
		\centering
		\includegraphics[width=0.9\linewidth]{figures/backward.jpg}
		\caption{后退LCD状态显示图}
		\label{backward}%文中引用该图片代号
	\end{minipage}
	%\qquad
	%让图片换行,
	
	\begin{minipage}{0.49\linewidth}
		\centering
		\includegraphics[width=0.9\linewidth]{figures/left.jpg}
		\caption{左转LCD状态显示图}
		\label{left}%文中引用该图片代号
	\end{minipage}
	\begin{minipage}{0.49\linewidth}
		\centering
		\includegraphics[width=0.9\linewidth]{figures/right.jpg}
		\caption{右转LCD状态显示图}
		\label{right}%文中引用该图片代号
	\end{minipage}
\end{figure}
\begin{figure}[!htbp]
    \centering
    \includegraphics[width =0.9\textwidth]{figures/stop.jpg}
    \caption{停止LCD状态显示图}
    \label{stop}
\end{figure}

\subsection{实物调试}
小车实物图如\autoref{小车实物图}所示:
\begin{figure}[!htbp]
    \centering
    \includegraphics[width =0.7\textwidth]{figures/小车实物正面图.jpg}
    \caption{小车实物图}
    \label{小车实物图}
\end{figure}
\newline
小车用手机蓝牙上位机软件控制,控制页面如\autoref{控制界面}所示,各控制命令设置如\autoref{前进命令设置}-\autoref{右转命令设置}所示:
\begin{figure}[!htbp]
    \centering
    \includegraphics[width =0.6\textwidth]{figures/控制界面.jpg}
    \caption{控制界面图}
    \label{控制界面}
\end{figure}
\begin{figure}[htbp]
	\centering
	\begin{minipage}{0.49\linewidth}
		\centering
		\includegraphics[width=0.6\linewidth]{figures/前进命令设置.jpg}
		\caption{前进命令设置图}
		\label{前进命令设置}%文中引用该图片代号
	\end{minipage}
	\begin{minipage}{0.49\linewidth}
		\centering
		\includegraphics[width=0.6\linewidth]{figures/后退命令设置.jpg}
		\caption{后退命令设置图}
		\label{后退命令设置}%文中引用该图片代号
	\end{minipage}
	%\qquad
	%让图片换行,
	
	\begin{minipage}{0.49\linewidth}
		\centering
		\includegraphics[width=0.6\linewidth]{figures/左转命令设置.jpg}
		\caption{左转命令设置图}
		\label{左转命令设置}%文中引用该图片代号
	\end{minipage}
	\begin{minipage}{0.49\linewidth}
		\centering
		\includegraphics[width=0.6\linewidth]{figures/右转命令设置.jpg}
		\caption{右转命令设置图}
		\label{右转命令设置}%文中引用该图片代号
	\end{minipage}
\end{figure}
\newline

小车上电后用手机蓝牙上位机控制,并成功实现所有预期功能,小车上电控制图如\autoref{小车上电控制图}所示:
\begin{figure}[!htbp]
    \centering
    \includegraphics[width =\textwidth]{figures/上电顶视图.jpg}
    \caption{小车上电控制图}
    \label{小车上电控制图}
\end{figure}
\section{课设问题分析}
\subsection{课设中的收获}
在完成本课设的过程中,我通过对小车硬件的学习和了解在一定程度上提高了我的硬件能力;还有在完成小车蓝牙通讯的过程中也丰富了我在单片机通讯方面的知识;在小车代码编写中,我使用c语言替代汇编语言进行代码编写,也提高了我在代码编写方面的能力;同时小车整体具备一定的复杂度,完成课设的过程中也提高了我解决复杂问题的能力。
\subsection{存在的问题}
由于小车硬件本身的局限性,该小车仅供学生学习,还不具备实用性,如果想要应用于生活中,还要改进很多东西。
\subsection{遇到的困难}
在从汇编语言转到c语言编写中,两种语言具有一定的异和同,需要一段时间去学习和适应。
\subsection{改进意见}
可以适当拓展其他的特定场所要求的功能开发,提高学生的即时开发能力。
\subsection{该技术的意义}
小车远程控制系统设计与实现,尤其是基于蓝牙模块或红外遥控的技术,对社会、健康、安全、法律以及文化等方面都有着不同程度的影响。下面是对这些方面影响的分析:
\subsubsection{社会影响}
\begin{itemize}
    \item 教育与娱乐:智能小车作为教育工具,能够激发儿童和青少年对科学、技术、工程和数学(STEM)的兴趣,促进动手能力和编程技能的培养。在娱乐领域,它们可以作为遥控玩具,增强家庭成员之间的互动。
    \item 智能家居与自动化:在智能家居环境中,智能小车可以被用作移动平台,执行监控、清洁或递送任务,提升居住舒适度和效率。
    \item 物流与工业:在物流和制造行业,智能小车可作为自动搬运工具,减少人力需求,提高生产率和供应链效率。
\end{itemize}

\subsubsection{健康影响}

\begin{itemize}
    \item 减少体力劳动:在物流和仓储领域,智能小车可以减少员工的体力负担,降低工作相关伤害的风险。
    \item 心理刺激与认知发展:对于儿童和老年人,操控智能小车可以作为一种脑部锻炼活动,有助于认知能力的保持和发展。
\end{itemize}

\subsubsection{安全影响}

\begin{itemize}
    \item 数据安全与隐私:蓝牙和红外遥控可能成为黑客攻击的目标,特别是当它们被用于更复杂的应用场景时,如家庭安全系统或工业控制系统。
    \item 物理安全:智能小车的不当使用可能导致物理伤害,尤其是在人口密集区域或儿童周围。
\end{itemize}

\subsubsection{法律影响}

\begin{itemize}
    \item 隐私保护法规:使用智能小车收集数据或监控的行为可能触犯隐私保护法律,特别是在未经同意的情况下。
    \item 产品责任:制造商和开发者需要确保智能小车的安全性,否则可能面临因产品缺陷导致的法律责任。
    \item 交通法规:如果智能小车在公共道路上使用,它们必须遵守相关的交通规则和标准,这可能涉及新的立法或现有法律的解释。
\end{itemize}

\subsubsection{文化影响}

\begin{itemize}
    \item 科技接受度:智能小车的普及反映了社会对新技术的接受程度,它们可能改变人们对自动化和人工智能的看法。
    \item 艺术与创意表达:在艺术和设计领域,智能小车可以成为创意作品的一部分,展示技术与美学的融合。
    \item 社会关系:随着智能小车在日常生活中扮演更多角色,它们可能影响人与人之间的交流,以及人与机器的关系。
\end{itemize}

综上所述,智能小车的远程控制系统不仅推动了技术进步,还对社会的多个层面产生了深远影响,需要平衡其带来的便利与潜在的风险。

\section{未来展望}
在结束我的项目报告之际,我对未来的研究方向和可能的技术进步充满期待。本项目—小车远程控制系统设计与实现,不仅展示了蓝牙模块和红外遥控技术在小型移动平台上的应用潜力,还通过LCD屏实现了对小车状态的实时反馈,为后续研究奠定了坚实的基础。

\subsection{集成AI与机器学习}

随着人工智能和机器学习技术的快速发展,未来的智能小车将不仅仅局限于远程控制,而是能够自主学习和适应环境。通过集成AI算法,小车可以实现更复杂的路径规划、障碍物检测和避免,甚至是在特定场景下的自我修复。这将极大地提升小车的实用性和安全性。

\subsection{增强现实(AR)与虚拟现实(VR)的应用}

将AR和VR技术融入小车控制系统,可以为用户提供更加沉浸式的操作体验。用户可以通过VR头盔直接观察到小车所处的真实环境,而AR则可以在现实世界中叠加虚拟信息,帮助用户更好地理解和控制小车。这种技术融合将使远程操作变得更加直观和高效。

\subsection{无线充电与能源管理}

无线充电技术的进步将解决智能小车的续航问题,使其能够在指定区域自动充电,延长工作时间。同时,高效的能源管理系统可以优化能量使用,减少能耗,提高小车的整体性能和使用寿命。

\subsection{物联网(IoT)与大数据分析}

通过连接物联网,智能小车可以成为更大网络中的一个节点,与其他设备进行数据交换和协作。大数据分析则可以帮助我从海量数据中提取有价值的信息,优化小车的设计和性能,同时也为预测性维护提供数据支持,进一步提升系统的可靠性和效率。

\subsection{安全性与隐私保护}

随着智能小车功能的增强和应用范围的扩大,安全性与隐私保护将成为重要议题。未来的研究需要关注如何加强数据加密,防止未授权访问,以及如何设计系统以保护用户的隐私,确保所有数据传输和存储的安全。
\newline

总之,智能小车远程控制系统的未来充满了无限可能。通过持续的技术创新和跨学科合作,我相信智能小车将逐步进化,成为日常生活和工业应用中不可或缺的一部分,为人类社会带来更多的便利和效率。

%%------------------------------参考文献---------------------------%%
\newpage
\reference
\addcontentsline{toc}{section}{参考文献}   % 将参考文献作为一级标题加入到目录

\end{document}